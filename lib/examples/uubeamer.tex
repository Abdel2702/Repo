\documentclass[10pt]{beamer}

% Needed for the pie chart below
\usepackage{pgf-pie}  


\usetheme[]{uubeamer}
% Use this option for a white title slide
% \usetheme[white]{uubeamer}

\title{Open blik, open houding, open wetenschap}

\author{Naam}
\institute{functie}

% Displayed on the title slide
\headline{EDIT SOURCE TO INSERT HEADLINE}

% Displayed in the footer on each slide after the title slide
\footline{Edit source to add footline}

% Shown on the disclaimer slide, e.g. to thank funding sources...
\disclaimer{Edit source to add disclaimer}

% You can add as many partnerlogos as you like. They will be re-scaled, grouped
% in threes, and put in a table on the disclaimers slide.
\partnerlogos{partner.png}{partner.png}{partner.png}{partner.png}{partner.png}%...

\date{8-11-2022}
% \date{8 November 2022}

\begin{document}

\frame{\titlepage}

\begin{frame}
  \frametitle{A sample slide\dots}

  Here are some bullet points:

  \begin{itemize}
    \item An item
    \item Another item
    \item Yet another item
    \item \dots
  \end{itemize}

  And here a number of important things:

  \begin{enumerate}
    \item Something important
    \item Something importanter
    \item \dots
  \end{enumerate}

\end{frame}

\begin{frame}
  \frametitle{Some math\dots}

\begin{theorem}[Rolle]
  Let $f:[a,b]\to\mathbb{R}$ be continuous, differiantable on $(a,b)$, and
  $f(a)=f(b)$. Then there exists a $c\in (a,b)$ such that:
  \[f'(c)=0\]
\end{theorem}

\begin{proof}
We consider two cases:

\begin{itemize}
  \item Case 1: $f(x)=0$ for all $x\in [a,b]$. The claim then follows since the
    derivatives of constant functions are always zero.
  \item Case 2: $f(x)\neq 0$ for some $x\in(a,b)$. By the EVT, $f$ takes its
    maximum and minimum value on $[a,b]$. \dots
\end{itemize}


\end{proof}


\end{frame}

\begin{frame}
  \frametitle{A pie chart with UU colors\dots}

  \begin{center}
    \begin{tikzpicture}

      \pie[color={uu-yellow,  uu-cream, uu-orange, uu-burgundy, uu-brown, uu-green,
      uu-blue, uu-dark-blue, uu-purple},
      text=legend, sum=auto, hide number]{%
        8.2/Category 1,
        3.2/Category 2,
        1.4/Category 3,
        1.2/Category 4,
        3.0/Category 5,
        2.0/Category 6,
        1.0/Category 7,
        0.5/Category 8,
        3.0/Category 9
      }

    \end{tikzpicture}
  \end{center}

\end{frame}

\end{document}
